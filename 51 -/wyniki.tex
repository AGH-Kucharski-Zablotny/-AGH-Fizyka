%!Mode:: "TeX:UTF-8"
\documentclass[a4paper,12pts]{article}

\usepackage[polish]{babel}
\usepackage[utf8]{inputenc}
\usepackage{fontspec}
\setmainfont{Calibri}

\linespread{1.15}

\usepackage{caption}
\captionsetup{%
	font={footnotesize},
	labelfont={bf}
}

\usepackage{anysize}
\usepackage{geometry}
\usepackage{multicol}
\usepackage{graphicx}

% Plik szablonowy do wykorzystania pózniej - nie zmieniaj go!

\begin{document}
	\thispagestyle{empty}
	\begin{flushleft}
		Wydział Elektrotechniki, Automatyki, Informatyki i Inżynierii Biomedycznej \\
		Informatyka, rok II \\
		Zespół numer 3 \\
		Piotr Kucharski \\
		Dominik Zabłotny \\
		\vspace*{\fill}
		%-----------NUMER CWICZENIA--------%
		{\large \textbf{Sprawozdanie z ćwiczenia nr 51} } \\
		%-----------TEMAT ĆWICZENIA--------%
		Cokolwiek....		
		\vfill	
		%-----------DATA-------------%
		15 listopada 2017r
	\end{flushleft}
	
	\newpage
	
	%---------------------------------------------------------------------------------------
	
	\section{Opracowanie wyników pomiarowych}
	Dla poszczególnych próbek dokonano wielokrotnych pomiarów, z których należy określić średni wynik. Dane dla płytki pleksiglasowej zostały przedstawione w tabeli \ref{gruby_pleksi}.
	
	% Wstaw tabelę tutaj
	
	SSSS
	
	Następnym badanym ciałem jest płytka pleksiglasowa o mniejszej grubości. Celem badania tego samego materiału jest określenie wpływu grubości ciała na współczynnik załamania światła w ośrodku. Zmierzone dane zawarto w tabeli \ref{chudy_pleksi}.
	
	% Wstaw tabelę chudego pleksi tutaj
	
	Aby porównać współczynnik załamania w innym ośrodku zbadano płytkę wykonaną ze szkła. Zmierzone dane przedstawiono w tabeli \ref{szklo}.
	
	% Wstaw tabelę dla szkła
	
		
	
\end{document}