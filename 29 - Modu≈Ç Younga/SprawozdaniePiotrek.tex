% !Mode:: "TeX:UTF-8"
\documentclass[a4paper,12pts]{article}

\usepackage[polish]{babel}
\usepackage{fontspec}
\setmainfont{Calibri}

\linespread{1.15}

% Plik szablonowy do wykorzystania pózniej - nie zmieniaj go!
% Użwyaj kompilatora XELATEX

\begin{document}
	\thispagestyle{empty}
	\begin{flushleft}
		Wydział Elektrotechniki, Automatyki, Informatyki i Inżynierii Biomedycznej \\
		Informatyka, rok II \\
		Zespół numer 3 \\
		Piotr Kucharski \\
		Dominik Zabłotny \\
		\vspace*{\fill}
		%-----------NUMER CWICZENIA--------%
		{\large \textbf{Sprawozdanie z ćwiczenia nr 29} } \\
		%-----------TEMAT ĆWICZENIA--------%
		Fale podłóżne w ciałach stałych.	
		\vfill	
		%-----------DATA-------------%
		18 października 2017r
	\end{flushleft}
	
	\newpage
	
%--------------------------------------------------------------------------------------------------------------
	
	\section{Wstęp}
	
	\subsection{Cele ćwiczenia}
	Celem ćwiczenia jest wyznaczenie modułu Younga dla prętów różnych materiałów na podstawie pomiarów ich częstotliwości harmonicznych.
	
	\subsection{Wprowadzenie teoretyczne}
	\subsubsection{Fala podłóżna}
	Fala podłóżna jest to fala powstająca przez gwałtowne wychylenie ciała z położenia równowagi oraz dalszemu jego drganiu aż do momentu odzyskania równowagi. Szybkość rozchodzenia się tej fali zależy od bezwładności i sprężystości ciała.
	
	\subsubsection{Moduł Younga}
	Wielkość charakteryzującą sprężystość materiału, będąca jego integralną częścią nazywamy modułem Younga oraz oznaczamy go jako $E$. Ogólny wzór na moduł Younga określa się jako stosunek naprężenia $\sigma$ do względnego odkszałcenia liniowego $\varepsilon$ materiału:
	\begin{equation}
		E = \frac{\sigma}{\varepsilon}
	\end{equation}
	Po uwzględnieniu, że ćwiczenie przeprowadzane jest na prętach materiałów, analizie rozchodzenia się fali podłóżej w pręcie oraz prawa Hooke'a uzyskujemy wzór:
	\begin{equation}
		E = 4 \rho l^2 f^2
	\end{equation}
	gdzie $\rho$ to gęstość materiału, $l$ - długość pręta oraz $f$ częstotliwość fali podłużnej. Tego wzoru będziemy używać do wykonania ćwiczenia.
	
	
	
%--------------------------------------------------------------------------------------------------------------
	
	\section{Wykonanie ćwiczenia}
	
%--------------------------------------------------------------------------------------------------------------
	
	\section{Opracowanie danych pomiarowych}
	
	%----------------------------------------------------------------------------------------------------------	
	
	\subsection{Analiza niepewności}
	
%--------------------------------------------------------------------------------------------------------------

	\section{Podsumowanie}

%--------------------------------------------------------------------------------------------------------------
	
\end{document}