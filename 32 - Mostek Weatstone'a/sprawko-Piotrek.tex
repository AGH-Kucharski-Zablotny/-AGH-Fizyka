%!Mode:: "TeX:UTF-8"
\documentclass[a4paper,12pts]{article}

\usepackage[polish]{babel}
\usepackage[utf8]{inputenc}
\usepackage{fontspec}
\setmainfont{Calibri}

\linespread{1.15}

\usepackage{caption}
\captionsetup{%
	font={footnotesize},
	labelfont={bf}
}

\usepackage{anysize}
\usepackage{geometry}

\usepackage{graphicx}

% Plik szablonowy do wykorzystania pózniej - nie zmieniaj go!

\begin{document}
	\thispagestyle{empty}
	\begin{flushleft}
		Wydział Elektrotechniki, Automatyki, Informatyki i Inżynierii Biomedycznej \\
		Informatyka, rok II \\
		Zespół numer 3 \\
		Piotr Kucharski \\
		Dominik Zabłotny \\
		\vspace*{\fill}
		%-----------NUMER CWICZENIA--------%
		{\large \textbf{Sprawozdanie z ćwiczenia nr 0} } \\
		%-----------TEMAT ĆWICZENIA--------%
		Wyznaczanie przyspieszenia ziemskiego za pomocą wahadła matematycznego.		
		\vfill	
		%-----------DATA-------------%
		11 października 2017r
	\end{flushleft}
	
	\newpage
	
%--------------------------------------------------------------------------------------------------------------
	
	\section{Cel ćwiczenia}
	
%--------------------------------------------------------------------------------------------------------------
	
	\section{Wykonanie ćwiczenia}
	
%--------------------------------------------------------------------------------------------------------------
	
	\section{Opracowanie danych pomiarowych}
	Długość druta oporowego została zmierzona i wynosi $100$ [cm]. Jest ona dana zmienną $l$
	\begin{equation}
		l = 100 \textrm{ cm}
	\end{equation}
	Rezystancję z danych podanych w tabelach obliczamy za pomocą wzoru \ref{wzor:opornosc}. Krok zmiany znanej rezystancji został dostosowany do danego opornika aby zmiana wychylenia na mikroamperomierza była zauważalna.
	
	\subsection{Pomiar dla opornika $R_1$}
	W tabeli \ref{tabela:opornik1} zestawiono pomiary przeprowadzone dla opornika 1. Przyjęty został krok zmiany znanej rezystancji $0.5 ~\Omega$.
	
	\begin{table}[!h]
		\centering
		\resizebox{\textwidth}{!}{\begin{tabular}{| c | c | c | c | c | c | c | c | c | c | c |}
			\hline
			Rezystancja opornika znanego [$\Omega$] & 12.5 & 13.0 & 13.5 & 14.0 & 14.5 & 12.0 & 11.5 & 11 & 10.5 & 10  \\ \hline
			Długość  $a$ [mm] & 500 & 491 & 482 & 473 & 464 & 510 & 519 & 529 & 543 & 555 \\ \hline
			Opór $R_1$ obliczona [$\Omega$] & 12.50 & 12.54 & 12.56 & 12.57 & 12.55 & 12.49 & 12.41 & 12.35 & 12.48 & 12.47 \\ \hline
		\end{tabular}}
		\caption{Wyniki pomiarów dla opornika nr 1}
		\label{tabela:opornik1}
	\end{table}

	Aby uzyskać rezystancję opornika obliczamy średnią arytmetyczną z wyników z tabeli powyżej:
	\begin{equation}
		\overline{R_1} = \frac{\sum_{i = 1}^{10} R_{1_i}}{10} \approx 12.49 \textrm{ $\Omega$}
	\end{equation}
	
	%----------------------------------------------------------------------------------------------------------	
	
	\subsection{Pomiar dla opornika $R_2$}
	W tabeli \ref{tabela:opornik2} zestawiono pomiary przeprowadzone dla opornika 2. Przyjęty został krok zmiany znanej rezystancji $1 ~\Omega$ z wyjątkiem pierwszego pomiaru dla $a = 500$ mm (celem uzyskania wyniku równego rezystancji znanej).
	
	\begin{table}[!h]
		\centering
		\resizebox{\textwidth}{!}{\begin{tabular}{| c | c | c | c | c | c | c | c | c | c | c |}
			\hline
			Rezystancja opornika znanego [$\Omega$] & 35.8 & 36.0 & 37.0 & 38.0 & 39.0 & 35.0 & 34.0 & 33.0 & 32.0 & 31.0  \\ \hline
			Długość  $a$ [mm] & 500 & 494 & 487 & 480 & 473 & 502 & 509 & 517 & 524 & 533 \\ \hline
			Opór $R_2$ obliczona [$\Omega$] & 35.80 & 35.15 & 35.12 & 35.08 & 35.00 & 35.28 & 35.25 & 35.32 & 35.23 & 35.38 \\ \hline
		\end{tabular}}
		\caption{Wyniki pomiarów dla opornika nr 2}
		\label{tabela:opornik2}
	\end{table}
	
	Aby uzyskać rezystancję opornika obliczamy średnią arytmetyczną z wyników z tabeli powyżej:
	\begin{equation}
	\overline{R_2} = \frac{\sum_{i = 1}^{10} R_{2_i}}{10} \approx 35.26 \textrm{ $\Omega$}
	\end{equation}
	
	%----------------------------------------------------------------------------------------------------------
	
	\subsection{Pomiar dla opornika $R_3$}
	W tabeli \ref{tabela:opornik3} zestawiono pomiary przeprowadzone dla opornika 3. Przyjęty został krok zmiany znanej rezystancji $2 ~\Omega$ z wyjątkiem pierwszego pomiaru dla $a = 500$ mm (celem uzyskania wyniku równego rezystancji znanej).
	
	\begin{table}[!h]
		\centering
		\resizebox{\textwidth}{!}{\begin{tabular}{| c | c | c | c | c | c | c | c | c | c | c |}
			\hline
			Rezystancja opornika znanego [$\Omega$] & 72.1 & 74.0 & 76.0 & 78.0 & 80.0 & 70.0 & 68.0 & 66.0 & 64.0 & 62.0  \\ \hline
			Długość  $a$ [mm] & 500 & 491 & 481 & 476 & 469 & 506 & 508 & 513 & 520 & 527 \\ \hline
			Opór $R_3$ obliczona [$\Omega$] & 72.10 & 71.38 & 70.44 & 70.85 & 70.66 & 71.70 & 70.21 & 69.52 & 69.33 & 69.08 \\ \hline
		\end{tabular}}
		\caption{Wyniki pomiarów dla opornika nr 3}
		\label{tabela:opornik3}
	\end{table}
	
	Aby uzyskać rezystancję opornika obliczamy średnią arytmetyczną z wyników z tabeli powyżej:
	\begin{equation}
	\overline{R_3} = \frac{\sum_{i = 1}^{10} R_{3_i}}{10} \approx 70.53 \textrm{ $\Omega$}
	\end{equation}
	
	%----------------------------------------------------------------------------------------------------------
	
	\subsection{Pomiar dla połączenia szeregowego}
	W tabeli \ref{tabela:szeregowe} zestawiono pomiary przeprowadzone połączenia szeregowego oporników $R_1$, $R_2$, $R_3$. Przyjęty został krok zmiany znanej rezystancji $5 ~\Omega$ z wyjątkiem pierwszego pomiaru dla $a = 500$ mm (celem uzyskania wyniku równego rezystancji znanej).
	
	\begin{table}[!h]
		\centering
		\resizebox{\textwidth}{!}{\begin{tabular}{| c | c | c | c | c | c | c | c | c | c | c |}
			\hline
			Rezystancja opornika znanego [$\Omega$] & 116.3 & 120.0 & 125.0 & 130.0 & 135.0 & 110.0 & 105.0 & 100.0 & 95.0 & 90  \\ \hline
			Długość  $a$ [mm] & 500 & 496 & 484 & 474 & 465 & 514 & 526 & 538 & 554 & 568 \\ \hline
			Opór $R_{s}$ obliczona [$\Omega$] & 116.30 & 118.10 & 117.25 & 117.15 & 117.34 & 116.34 & 116.52 & 116.45 & 118.00 & 118.33 \\ \hline
		\end{tabular}}
		\caption{Wyniki pomiarów dla połączenia szeregowego}
		\label{tabela:szeregowe}
	\end{table}
	
	Aby uzyskać rezystancję opornika obliczamy średnią arytmetyczną z wyników z tabeli powyżej:
	\begin{equation}
	\overline{R_{s}} = \frac{\sum_{i = 1}^{10} R_{{s}_i}}{10} \approx 117.18 \textrm{ $\Omega$}
	\end{equation}
	
	%----------------------------------------------------------------------------------------------------------
	
	\subsection{Pomiar dla połączenia szeregowego}
	W tabeli \ref{tabela:rownolegle} zestawiono pomiary przeprowadzone połączenia równoległego oporników $R_1$, $R_2$, $R_3$. Przyjęty został krok zmiany znanej rezystancji $0.5 ~\Omega$ z wyjątkiem pierwszego pomiaru dla $a = 500$ mm (celem uzyskania wyniku równego rezystancji znanej).
	
	\begin{table}[!h]
		\centering
		\resizebox{\textwidth}{!}{\begin{tabular}{| c | c | c | c | c | c | c | c | c | c | c |}
				\hline
				Rezystancja opornika znanego [$\Omega$] & 8.0 & 8.5 & 9.0 & 9.5 & 10.0 & 7.5 & 7.0 & 6.5 & 6.0 & 5.5  \\ \hline
				Długość  $a$ [mm] & 500 & 485 & 472 & 459 & 450 & 519 & 534 & 552 & 557 & 596 \\ \hline
				Opór $R_{r}$ obliczona [$\Omega$] & 8.0 & 8.0 & 8.05 & 8.06 & 8.18 & 8.09 & 8.02 & 8.01 & 7.95 & 8.11 \\ \hline
		\end{tabular}}
		\caption{Wyniki pomiarów dla połączenia szeregowego}
		\label{tabela:rownolegle}
	\end{table}
	
	Aby uzyskać rezystancję opornika obliczamy średnią arytmetyczną z wyników z tabeli powyżej:
	\begin{equation}
	\overline{R_{r}} = \frac{\sum_{i = 1}^{10} R_{{r}_i}}{10} \approx 8.05 \textrm{ $\Omega$}
	\end{equation}
	
	%----------------------------------------------------------------------------------------------------------
	
	\subsection{Analiza niepewności}
	
%--------------------------------------------------------------------------------------------------------------

	\section{Podsumowanie}

%--------------------------------------------------------------------------------------------------------------

\end{document}