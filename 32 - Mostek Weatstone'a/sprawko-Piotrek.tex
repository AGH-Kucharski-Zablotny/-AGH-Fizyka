%!Mode:: "TeX:UTF-8"
\documentclass[a4paper,12pts]{article}

\usepackage[polish]{babel}
\usepackage[utf8]{inputenc}
\usepackage{fontspec}
\setmainfont{Calibri}

\linespread{1.15}

\usepackage{caption}
\captionsetup{%
	font={footnotesize},
	labelfont={bf}
}

\usepackage{anysize}
\usepackage{geometry}

\usepackage{graphicx}

% Plik szablonowy do wykorzystania pózniej - nie zmieniaj go!

\begin{document}
	\thispagestyle{empty}
	\begin{flushleft}
		Wydział Elektrotechniki, Automatyki, Informatyki i Inżynierii Biomedycznej \\
		Informatyka, rok II \\
		Zespół numer 3 \\
		Piotr Kucharski \\
		Dominik Zabłotny \\
		\vspace*{\fill}
		%-----------NUMER CWICZENIA--------%
		{\large \textbf{Sprawozdanie z ćwiczenia nr 32} } \\
		%-----------TEMAT ĆWICZENIA--------%
		Mostek Wheatstone'a		
		\vfill	
		%-----------DATA-------------%
		25 października 2017r
	\end{flushleft}
	
	\newpage
	
%--------------------------------------------------------------------------------------------------------------
	
	\section{Cel ćwiczenia}
	
%--------------------------------------------------------------------------------------------------------------
	
	\section{Wykonanie ćwiczenia}
	
%--------------------------------------------------------------------------------------------------------------
	
	\section{Opracowanie danych pomiarowych}
	Długość druta oporowego została zmierzona i wynosi $100$ cm. Jest ona dana zmienną $l$
	\begin{equation}
		l = 100 \textrm{ cm}
	\end{equation}
	Rezystancja z danych podanych w tabelach zostaje obliczona za pomocą wzoru \ref{wzor:opornosc}. Krok zmiany znanej rezystancji został dostosowany do danego opornika aby zmiana wychylenia mikroamperomierza była zauważalna.
	
	\subsection{Pomiar dla opornika 1}
	W tabeli \ref{tabela:opornik1} zestawiono pomiary przeprowadzone dla opornika 1. Przyjęty został krok zmiany znanej rezystancji $0.5 ~\Omega$.
	
	\begin{table}[!h]
		\centering
		\resizebox{\textwidth}{!}{\begin{tabular}{| c | c | c | c | c | c | c | c | c | c | c |}
			\hline
			Rezystancja opornika znanego $R$ [$\Omega$] & 12.5 & 13.0 & 13.5 & 14.0 & 14.5 & 12.0 & 11.5 & 11 & 10.5 & 10  \\ \hline
			Długość  $a$ [mm] & 500 & 491 & 482 & 473 & 464 & 510 & 519 & 529 & 543 & 555 \\ \hline
			Opór $R_1$ obliczona [$\Omega$] & 12.50 & 12.54 & 12.56 & 12.57 & 12.55 & 12.49 & 12.41 & 12.35 & 12.48 & 12.47 \\ \hline
		\end{tabular}}
		\caption{Wyniki pomiarów dla opornika nr 1}
		\label{tabela:opornik1}
	\end{table}

	Aby uzyskać rezystancję opornika obliczamy średnią arytmetyczną z wyników z tabeli powyżej:
	\begin{equation}
		\overline{R_1} = \frac{\sum_{i = 1}^{10} R_{1_i}}{10} \approx 12.49 \textrm{ $\Omega$}
	\end{equation}
	
	%----------------------------------------------------------------------------------------------------------	
	
	\subsection{Pomiar dla opornika 2}
	W tabeli \ref{tabela:opornik2} zestawiono pomiary przeprowadzone dla opornika 2. Przyjęty został krok zmiany znanej rezystancji $1 ~\Omega$ z wyjątkiem pierwszego pomiaru dla $a = 500$ mm (celem uzyskania wyniku równego rezystancji znanej).
	
	\begin{table}[!h]
		\centering
		\resizebox{\textwidth}{!}{\begin{tabular}{| c | c | c | c | c | c | c | c | c | c | c |}
			\hline
			Rezystancja opornika znanego $R$ [$\Omega$] & 35.8 & 36.0 & 37.0 & 38.0 & 39.0 & 35.0 & 34.0 & 33.0 & 32.0 & 31.0  \\ \hline
			Długość  $a$ [mm] & 500 & 494 & 487 & 480 & 473 & 502 & 509 & 517 & 524 & 533 \\ \hline
			Opór $R_2$ obliczona [$\Omega$] & 35.80 & 35.15 & 35.12 & 35.08 & 35.00 & 35.28 & 35.25 & 35.32 & 35.23 & 35.38 \\ \hline
		\end{tabular}}
		\caption{Wyniki pomiarów dla opornika nr 2}
		\label{tabela:opornik2}
	\end{table}
	
	Aby uzyskać rezystancję opornika obliczamy średnią arytmetyczną z wyników z tabeli powyżej:
	\begin{equation}
	\overline{R_2} = \frac{\sum_{i = 1}^{10} R_{2_i}}{10} \approx 35.26 \textrm{ $\Omega$}
	\end{equation}
	
	%----------------------------------------------------------------------------------------------------------
	
	\subsection{Pomiar dla opornika 3}
	W tabeli \ref{tabela:opornik3} zestawiono pomiary przeprowadzone dla opornika 3. Przyjęty został krok zmiany znanej rezystancji $2 ~\Omega$ z wyjątkiem pierwszego pomiaru dla $a = 500$ mm (celem uzyskania wyniku równego rezystancji znanej).
	
	\begin{table}[!h]
		\centering
		\resizebox{\textwidth}{!}{\begin{tabular}{| c | c | c | c | c | c | c | c | c | c | c |}
			\hline
			Rezystancja opornika znanego $R$ [$\Omega$] & 72.1 & 74.0 & 76.0 & 78.0 & 80.0 & 70.0 & 68.0 & 66.0 & 64.0 & 62.0  \\ \hline
			Długość  $a$ [mm] & 500 & 491 & 481 & 476 & 469 & 506 & 508 & 513 & 520 & 527 \\ \hline
			Opór $R_3$ obliczona [$\Omega$] & 72.10 & 71.38 & 70.44 & 70.85 & 70.66 & 71.70 & 70.21 & 69.52 & 69.33 & 69.08 \\ \hline
		\end{tabular}}
		\caption{Wyniki pomiarów dla opornika nr 3}
		\label{tabela:opornik3}
	\end{table}
	
	Aby uzyskać rezystancję opornika obliczamy średnią arytmetyczną z wyników z tabeli powyżej:
	\begin{equation}
	\overline{R_3} = \frac{\sum_{i = 1}^{10} R_{3_i}}{10} \approx 70.53 \textrm{ $\Omega$}
	\end{equation}
	
	%----------------------------------------------------------------------------------------------------------
	
	\subsection{Połączenie szeregowe}
	\subsubsection{Obliczenie rezystancji}
	W połaczeniu szeregowym opór zastępczy jest równy sumie oporów rezystorów na połączeniu, stąd wynika, że mierzony opór powinien wynosić:
	\begin{equation}
		R_{s_{obl}} = R_1 + R_2 + R_3 = 118.28~\Omega
	\end{equation}
	
	\subsubsection{Pomiar rezystancji}
	W tabeli \ref{tabela:szeregowe} zestawiono pomiary przeprowadzone połączenia szeregowego oporników $R_1$, $R_2$, $R_3$. Przyjęty został krok zmiany znanej rezystancji $5 ~\Omega$ z wyjątkiem pierwszego pomiaru dla $a = 500$ mm (celem uzyskania wyniku równego rezystancji znanej).
	
	\begin{table}[!h]
		\centering
		\resizebox{\textwidth}{!}{\begin{tabular}{| c | c | c | c | c | c | c | c | c | c | c |}
			\hline
			Rezystancja opornika znanego $R$ [$\Omega$] & 116.3 & 120.0 & 125.0 & 130.0 & 135.0 & 110.0 & 105.0 & 100.0 & 95.0 & 90  \\ \hline
			Długość  $a$ [mm] & 500 & 496 & 484 & 474 & 465 & 514 & 526 & 538 & 554 & 568 \\ \hline
			Opór $R_{s}$ obliczona [$\Omega$] & 116.30 & 118.10 & 117.25 & 117.15 & 117.34 & 116.34 & 116.52 & 116.45 & 118.00 & 118.33 \\ \hline
		\end{tabular}}
		\caption{Wyniki pomiarów dla połączenia szeregowego}
		\label{tabela:szeregowe}
	\end{table}
	
	Aby uzyskać rezystancję opornika obliczamy średnią arytmetyczną z wyników z tabeli powyżej:
	\begin{equation}
	\overline{R_{s}} = \frac{\sum_{i = 1}^{10} R_{{s}_i}}{10} \approx 117.18 \textrm{ $\Omega$}
	\end{equation}
	
	%----------------------------------------------------------------------------------------------------------
	
	\subsection{Połączenie równoległe}
	\subsubsection{Obliczenie rezystancji}
	W połaczeniu równoległym opór zastępczy jest równy odwrotności sumy odwrotności oporów rezysotrów układu. Wynika stąd, że spodziewana oporność połączenia będzie wynosić:
	\begin{equation}
		R_{r_{obl}} = \frac{1}{\frac{1}{R_1} + \frac{1}{R_2} + \frac{1}{R_3}} \approx 8.16~\Omega 
	\end{equation}
	
	\subsubsection{Pomiar rezystancji}
	W tabeli \ref{tabela:rownolegle} zestawiono pomiary przeprowadzone połączenia równoległego oporników $R_1$, $R_2$, $R_3$. Przyjęty został krok zmiany znanej rezystancji $0.5 ~\Omega$ z wyjątkiem pierwszego pomiaru dla $a = 500$ mm (celem uzyskania wyniku równego rezystancji znanej).
	
	\begin{table}[!h]
		\centering
		\resizebox{\textwidth}{!}{\begin{tabular}{| c | c | c | c | c | c | c | c | c | c | c |}
				\hline
				Rezystancja opornika znanego $R$ [$\Omega$] & 8.0 & 8.5 & 9.0 & 9.5 & 10.0 & 7.5 & 7.0 & 6.5 & 6.0 & 5.5  \\ \hline
				Długość  $a$ [mm] & 500 & 485 & 472 & 459 & 450 & 519 & 534 & 552 & 557 & 596 \\ \hline
				Opór $R_{r}$ obliczona [$\Omega$] & 8.0 & 8.0 & 8.05 & 8.06 & 8.18 & 8.09 & 8.02 & 8.01 & 7.95 & 8.11 \\ \hline
		\end{tabular}}
		\caption{Wyniki pomiarów dla połączenia równoległego}
		\label{tabela:rownolegle}
	\end{table}
	
	Aby uzyskać rezystancję opornika obliczamy średnią arytmetyczną z wyników z tabeli powyżej:
	\begin{equation}
	\overline{R_{r}} = \frac{\sum_{i = 1}^{10} R_{{r}_i}}{10} \approx 8.05 \textrm{ $\Omega$}
	\end{equation}
	
	%----------------------------------------------------------------------------------------------------------
	
	\subsection{Analiza niepewności}
	
	\subsubsection{Niepewność pomiarowa długości druta oporowego}
	
	Drut oporowy został rozciagnięty nad całą długością przymiaru milimetrowego, dlatego stosujemy niepewność typu B, czyli wartość działki elementarnej:
	\begin{equation}
		u(l) = 0.1 \textrm{ cm}
	\end{equation}
	
	%----------------------------------------------------------------------------------------------------------
	
	\subsubsection{Niepewność pomiarowa opornika o znanej rezystancji}
	
	Oporność rezystora jest ściśle określona oraz możliwa do regulacji pokrętłami na obudowie urządzenia. Niestety, urządzenie nie oferowało żadnej podanej niepewności na tabliczce znamionowej, przez co należy przyjąć, że ustawiona rezystancja nie jest obarczona żadnym błędem. Jest to nieprawdziwe, ponieważ po sprawdzeniu wartości oporu za pomocą multimetru ustawiona wartość odbiegała od rzeczywistej o niestały procent, przez co nie jest możliwe określić dokładnej niepewności pomiarowej znanej rezystancji.
	
	%----------------------------------------------------------------------------------------------------------
	
	\subsubsection{Niepewność mierzonego oporu}
	
	Niepewność mierzonego oporu należy obliczyć zgodnie z prawem przenoszenia niepewności uzależnione od niepewności pomiarowej druta oporowego zgodnie z wzorem:
	\begin{equation}
		u(R_{{x}_i}) = \left| \frac{dR_{{x}_i}}{da} u(a) \right| = R \frac{l}{(l-a)^2} u(a)
	\end{equation}
	Powyższy wzór został zastosowany dla każdego pojedynczego pomiaru każdego opornika/zestawu oporników a z otrzymanych wyników została obliczona średnia arytmetyczna stanowiąca finalny wynik niepewności pomiarowej mierzonego oporu. Zatem wzór końcowy niepewności określony jest równaniem:
	\begin{equation}
		u(R_x) = \frac{\sum_{i = 1}^{10} R_{{x}_i}}{10}
	\end{equation}
	Niepewność wynosi odpowiednio dla poszczególnych rezystorów:
	\begin{itemize}
		\item Opornik 1
			\begin{equation}
				u(R_1) = 0.05~\Omega
			\end{equation}
		
		\item Opornik 2
			\begin{equation}
				u(R_2) = 0.14~\Omega
			\end{equation}
			
		\item Opornik 3
			\begin{equation}
				u(R_3) = 0.28~\Omega
			\end{equation}
			
		\item Połaczenie szeregowe
			\begin{equation}
				u(R_s) = 0.47~\Omega
			\end{equation}
			
		\item Połaczenie równoległe
			\begin{equation}
				u(R_r) = 0.03~\Omega
			\end{equation}
	\end{itemize}

	\subsubsection{Niepewność złożona obliczonej rezystancji}
	Niepewność obliczenia rezystancji jest niepewnością złożoną uzależnioną od niepewności pomiaru oporności $R_1$, $R_2$ oraz $R_3$ i wyraża się ją wzorem:
	\begin{equation}
		u(R) = \sqrt{\left( \frac{\delta R}{\delta R_1} \right)^2 u(R_1)^2 + \left( \frac{\delta R}{\delta R_2} \right)^2 u(R_2)^2 + \left( \frac{\delta R}{\delta R_3} \right)^2 u(R_3)^2}
		\label{niepewnosc_zlozona}
	\end{equation}
	Stąd dla połączenia szeregowego uzyskujemy niepewność określoną wzorem:
	\begin{equation}
		u(R_{s_{obl}}) = \sqrt{u(R_1)^2 + u(R_2)^2 + u(R_3)^2} \approx 0.32~\Omega
	\end{equation}
	Dla połączenia równoległego niepewność złożona, po podstawieniu do wzoru \ref{niepewnosc_zlozona}, wynosi:
	\begin{equation}
		u(R_{r_{obl}}) \approx 0.02~\Omega
	\end{equation}
%--------------------------------------------------------------------------------------------------------------

	\section{Podsumowanie}
	Wyniki z uwzględnieniem niepewności pomiarowych zostały przedstawione w tabeli poniżej. Dla oporników 1, 2 oraz 3 wartość rezystancji jest wyznaczana tylko doświadczalnie.
	\begin{table}[!h]
		\centering
		\begin{tabular}{| c | c | c |}
			\cline{1-2}
			\textbf{Opornik} & \textbf{Oporność wyznaczona [$\Omega$]} \\ \cline{1-2}
			Opornik 1 & $12.49 \pm 0.05$ \\ \cline{1-2}
			Opornik 2 & $35.26 \pm 0.14$ \\ \hline
			Opornik 3 & $70.53 \pm 0.28$ & \textbf{Oporność obliczona [$\Omega$]} \\ \hline
			Połączenie szeregowe & $117.18 \pm 0.47$ & $118.28 \pm 0.32$\\ \hline
			Połączenie równoległe & $8.05 \pm 0.03$ & $8.16 \pm 0.02$\\ \hline
		\end{tabular}
		\caption{Podsumowanie wyników doświadczenia wraz z niepewnościami}
	\end{table}

%--------------------------------------------------------------------------------------------------------------

\end{document}