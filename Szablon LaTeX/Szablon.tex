%!Mode:: "TeX:UTF-8"
\documentclass[a4paper,12pts]{article}

\usepackage[polish]{babel}
\usepackage[utf8]{inputenc}
\usepackage{fontspec}
\setmainfont{Calibri}

\linespread{1.15}

\usepackage{caption}
\captionsetup{%
	font={footnotesize},
	labelfont={bf}
}

\usepackage{anysize}
\usepackage{geometry}

\usepackage{graphicx}

% Plik szablonowy do wykorzystania pózniej - nie zmieniaj go!

\begin{document}
	\thispagestyle{empty}
	\begin{flushleft}
		Wydział Elektrotechniki, Automatyki, Informatyki i Inżynierii Biomedycznej \\
		Informatyka, rok II \\
		Zespół numer 3 \\
		Piotr Kucharski \\
		Dominik Zabłotny \\
		\vspace*{\fill}
		%-----------NUMER CWICZENIA--------%
		{\large \textbf{Sprawozdanie z ćwiczenia nr 0} } \\
		%-----------TEMAT ĆWICZENIA--------%
		Wyznaczanie przyspieszenia ziemskiego za pomocą wahadła matematycznego.		
		\vfill	
		%-----------DATA-------------%
		11 października 2017r
	\end{flushleft}
	
	\newpage
	
%--------------------------------------------------------------------------------------------------------------
	
	\section{Cel ćwiczenia}
	
%--------------------------------------------------------------------------------------------------------------
	
	\section{Wykonanie ćwiczenia}
	
%--------------------------------------------------------------------------------------------------------------
	
	\section{Opracowanie danych pomiarowych}
	
	%----------------------------------------------------------------------------------------------------------	
	
	\subsection{Analiza niepewności}
	
%--------------------------------------------------------------------------------------------------------------

	\section{Podsumowanie}

%--------------------------------------------------------------------------------------------------------------

\end{document}