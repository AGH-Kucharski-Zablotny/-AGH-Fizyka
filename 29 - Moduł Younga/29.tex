%!Mode:: "TeX:UTF-8"
\documentclass[a4paper,12pts]{article}

\usepackage[polish]{babel}
\usepackage[utf8]{inputenc}
\usepackage{fontspec}
\setmainfont{Calibri}

\linespread{1.15}

\usepackage{caption}
\captionsetup{%
	font={footnotesize},
	labelfont={bf}
}

\usepackage{anysize}
\usepackage{geometry}

\usepackage{graphicx}


% Plik szablonowy do wykorzystania pózniej - nie zmieniaj go!
% Użwyaj kompilatora XELATEX

\begin{document}
	\thispagestyle{empty}
	\begin{flushleft}
		Wydział Elektrotechniki, Automatyki, Informatyki i Inżynierii Biomedycznej \\
		Informatyka, rok II \\
		Zespół numer 3 \\
		Piotr Kucharski \\
		Dominik Zabłotny \\
		\vspace*{\fill}
		%-----------NUMER CWICZENIA--------%
		{\large \textbf{Sprawozdanie z ćwiczenia nr 29} } \\
		%-----------TEMAT ĆWICZENIA--------%
		Fale podłóżne w ciałach stałych.	
		\vfill	
		%-----------DATA-------------%
		18 października 2017r
	\end{flushleft}
	
	\newpage
	
%--------------------------------------------------------------------------------------------------------------

\section{Wstęp}

\subsection{Cele ćwiczenia}
Celem ćwiczenia jest wyznaczenie modułu Younga dla prętów różnych materiałów na podstawie pomiarów ich częstotliwości harmonicznych.

\subsection{Wprowadzenie teoretyczne}
\subsubsection{Fala podłużna}
Fala podłóżna jest to fala powstająca przez gwałtowne wychylenie ciała z położenia równowagi oraz dalszemu jego drganiu aż do momentu odzyskania równowagi. Szybkość rozchodzenia się tej fali zależy od bezwładności i sprężystości ciała.

\subsection{Fala stojąca}
Fala, której grzbiety i doliny nie przemieszaczją się. Powstaje na skutek interferencji dwóch takich samych fal podłużnych, w przypadku pręta przez odbicie fali o przeciwległy koniec do uderzonego młotkiem. Falę stojącą określa się równaniem:
\begin{equation}
y(x) = 2A \sin kx
\end{equation}
gdzie $A$ to amplituda fali wynikającej z uderzenia.

\subsubsection{Moduł Younga}
Nazywana również modułem odkształcalności liniowej - wielkość charakteryzującą sprężystość materiału, będąca jego integralną częścią nazywamy modułem Younga oraz oznaczamy go jako $E$. Ogólny wzór na moduł Younga określa się jako stosunek naprężenia $\sigma$ do względnego odkszałcenia liniowego $\varepsilon$ materiału:
\begin{equation}
E = \frac{\sigma}{\varepsilon}
\end{equation}
Po uwzględnieniu ruchu fali w pręcie, skorzystaniu z prawa Hooke'a i przekształceniach otrzymujemy, że moduł Younga jest zależny od gęstości ciała i kwadratu prędkości rozchodzenia sie fali w nim, co przedstawia wzór:
\begin{equation}
	E = \rho v^2
\end{equation}
co po uzględnieniu, że fala padająca i odbita interferują ze sobą daje:
\begin{equation}
E = 4 \rho l^2 f^2
\end{equation}
gdzie $\rho$ to gęstość materiału, $l$ - odległość między węzłami fali oraz $f$ częstotliwość fali podłużnej. Tego wzoru będziemy używać do wykonania ćwiczenia.

\subsubsection{Interferencja fal}
Jest to zjawisko ,,łączenia się"~dwóch lub więcej fal tej samej częstotliwości w wyniku ich superpozycji. Efektem jest powstanie fali o nowej amplitudzie, lecz tej samej częstotliwości - wszystkie fale łączą się w jedną. Warunkiem potrzebnym do zajścia interferencji jest spójność fal (korelacja faz i równość częstotliwości). Matematycznie falę taką określamy wzorem:
\begin{equation}
	y(t) = A\sin (\omega t + \phi_1) + A \sin (\omega t + \phi_2)
\end{equation}
gdzie $A$ to amplituda fal, $\phi_1$, $\phi_2$ - fazy początkowe.

\subsubsection{Analiza Fouriera i szybka transformata Fouriera}
Jest to proces badania drgań harmonicznych, polega na przedstawieniu funkcji okresowej w postaci nieskończonego szeregu trygonometrycznego (szeregu Fouriera). W naszym doświadczeniu wykorzystujemy program Zelscope, który realizuje algorytm FFT (obliczający transformatę Fouriera) i przedstawia ją jako widmo fali na ekranie (analogicznie do oscyloskopu znanego z elektroniki). Szybka transformata Fouriera to algorytm obliczający dyskretną transformatę Fouriera określoną wzorem:
\begin{equation}
	X_k = \sum_{n=0}^{N-1} x_ne^{-\frac{2 \pi i}{N}nk}
\end{equation} 
dla $k = 0, ..., N-1$. Będziemy odczytywać kolejne wartości drgań harmonicznych z ekranu.

\subsubsection{Niepewności pomiarowe}
Podczas naszego doświadczenia będziemy analizować wyniki z niepewnościami pomiarowymi typu A (związane z niedokładnością przyrządów pomiarowych, oczytywana ze specyfikacji produktu) oraz B (spowodowane wieloma pomiarami, które każde obarczone jest błędem). Są one spowodowane hałasem otoczenia podczas wykonywania pomiaru widma fali dźwiękowej, wielokrotnym pomiarom tej wielkości, złym refleksem osoby zamrażającej pomiar programu Zelscope oraz niedokładnością przyrządów pomiarowych takich jak suwmiara, miara zwijana czy też waga. Wiele nieudanych prób pozyskania widma fali harmonicznej została pominięta, ponieważ nie wnoszą one niczego do wyników doświadczenia. Dodatkowym czynnikiem wpływającym na niedokładność wyników jest wykorzystanie ciał nie wykonanych całkowicie z danego materiału, lecz zawierających pewne czynniki zaburzające, np. naklejkę. Następną obliczaną niepewnością jest niepewność złożona z racji obliczania kolejnych wartości z wielkości, które już są obarczone pewnym błędem. Wszelkie wzory zostaną przedstawione w rozdziale 3.2


\subsection{Układ pomiarowy}
Układ pomiarowy składa się z komputera z zainstalowanym oprogramowaniem Zelscope, mikrofonu podłączonego do komuptera, długich prętów wykonanych z różnych materiałów zawieszonych na nitkach w dwóch miejscach. Do wprawienia ciał w drgania użyjemy młotka, do pomiaru długości prętów użyjemy miary milimetrowej zwijanej, do pomiaru masy prętów użyjemy wagi elektronicznej firmy RAWAG model WTB 200 oraz do zmierzenia grubości materiałów w celu wyznaczenia ich objętości użyjemy suwmiarki.

\begin{figure}[!h]
	\centering
	\includegraphics[scale=0.5]{schemat}
	\caption{Schemat układu pomiarowego \\ \textit{źródło: rysunek autorski} }
	\label{schematUkladu}
\end{figure}

%--------------------------------------------------------------------------------------------------------------

\newpage \section{Wykonanie ćwiczenia}
Wykonanie ćwiczenia dzieli się na dwa kroki stosowane dla każdego badanego pręta oraz jednej wspólnej analizy wyników.
\subsection{Pomiary specyfikacji prętów}
\begin{itemize}
	\item Pomiar długości pręta za pomocą miary zwijanej.
	\item Pomiar grubości pręta za pomocą suwmiarki (w przypadku otwartego walca mierzymy promień zewnętrzny i wewnętrzny).
	\item Zważenie pręta w najlepszy możliwy sposób za pomocą wagi elektronicznej.
	\item Zapisanie wyników o danym ciele do tabeli.
\end{itemize}

\subsection{Pomiar częstotliwości harmonicznych}
\begin{itemize}
	\item Osadzenie preta w niciach zamontowanych do stelaża.
	\item Przybliżenie mikrofonu do badanego pręta.
	\item Uderzenie młotkiem w pręt aby wprawić go w drganie.
	\item Zamrożenie odczytu programu Zelscope w momencie najlepszej widoczności widma fal harmonicznych.
	\item Odczyt sześciu pierwszych harmonicznych (jeżeli taką ilość udało się zaobserwować).
\end{itemize}
W przypadku niejednoznacznego odczytu częstotliwości harmonicznych nalezy powtórzyć pomiar.

\newpage
\subsection{Oblicznanie koniecznych wartości}
Z zapisanych danych pomiarowych należy obliczyć gęstość ciała daną wzorem:
\begin{equation}
\rho = \frac{m}{V}
\end{equation}
gdzie $m$ to zmierzona masa ciała oraz $V$ to objętość ciała obliczona odpowiednio dla każdego pręta z odpowiednich wielkości. Pręty są różnymi figurami przestrzennymi, przez co wykorzystujemy odpowiedni wzór dla:
\begin{itemize}
	\item walca o promieniu podstawy $r$ oraz wysokości $h$
	\begin{equation}
	V = \pi r^2 h
	\end{equation}
	
	\item prostopadłościanu prawidłowego czworokątnego o krawędzi podstawy $a$ oraz wysokości $h$
	\begin{equation}
	V = a^2 h
	\end{equation}
	
	\item otwartego walca o promieniu zewnętrznym podstawy $R$, promieniu wewnętrznym podstawy $r$ oraz wysokości $h$
	\begin{equation}
	V = \pi (R^2 - r^2) h
	\end{equation}
\end{itemize}
Do oblicznia długości fali $\lambda$ zastosujemy zależność stosunku długości pręta $L$ do numeru harmonicznej fali $n$:
\begin{equation}
\lambda = \frac{2L}{n}
\end{equation}
Odległość $l$ między węzłami fali stojącej stanowi połowę jej długości:
\begin{equation}
l = \frac{1}{2} \lambda
\end{equation}
Do obliczenia predkości rozchodzenia się fali zastosujemy wzór:
\begin{equation}
V = 2lf
\end{equation}
\newpage
%--------------------------------------------------------------------------------------------------------------
	
	\section{Opracowanie danych pomiarowych}
	
	%----------------------------------------------------------------------------------------------------------	
	
	\subsection{Wyniki pomiarów}
	
	Zmierzone wielkości próbek zostały zapisane w tabeli 1, gdzie zapisano również wyniki wyliczonych wartości obliczonych za pomocą wzorów (7), (8), (9), (10) i (13). W kolejnych tabelach przestawione zostaną wyniki pomiarów zarejestrowanych częstotliwości dla kolejnych harmonicznych każdego materiału.
	
		\begin{table}[h!]
		\centering
		\begin{tabular}{ | c | c | c | c | c | c | }
			
			\hline
			\textrm{Materiał} & \textrm{Kształt} & \textrm{Masa [kg]} & \textrm{Długość [m]} & \textrm{Objętość [m$^2$]} & \textrm{Gęstość [kg/m$^3$]} \\ \hline
			\textrm{Aluminium} & \textrm{Walec} & 0.030 & 0.561 & 1.102 $\cdot$ 10$^{-5}$ & 2720.326 \\ \hline
			\textrm{Mosiądz} & \textrm{Walec} & 0.237 & 0.998 & 2.821 $\cdot$ 10$^{-5}$ & 8401.275 \\ \hline
			\textrm{Stal} & \textrm{Prostopadłościan} & 2.794 & 1.802 & 3.710 $\cdot$ 10$^{-4}$ & 7529.529 \\ \hline
			\textrm{Stal} & \textrm{Walec} & 1.138 & 1.800 & 1.470 $\cdot$ 10$^{-4}$ & 7741.496 \\ \hline
			\textrm{Żeliwo szare} & \textrm{Walec otwarty} & 0.760 & 1.800 & 1.095$\cdot$ 10$^{-4}$ & 6940.639 \\ \hline
			
		\end{tabular}
		\caption{Dane pomiarowe dla pięciu próbek}
		\label{Tabela1}	
	\end{table}


	\begin{table}[h!]
	\centering
		\begin{tabular}{|c|c|c|c|}
			
		\cline{1-2}
	\multicolumn{2}{|l|}{\begin{tabular}{c} \textrm{Analiza Aluminium} $l = 0.561$ [m]\end{tabular}} & \multicolumn{2}{c}{}\\
	\hline
	\textrm{Nr harmonicznej} & \textrm{Częstotliwość} $f$ \textrm{[Hz]} & \textrm{Długość fali} $ \lambda$ \textrm{[m]} & \textrm{Prędkość fali} $\upsilon$ \textrm{[m/s]}  \\ 
		\hline
		1 & 2411.76 & 1.122 & 2705.994 \\ \hline
		2  & 4941.18 & 0.561 & 2772.001 \\ \hline
		3 & 7382.35  & 0.374 & 2760.998 \\ \hline
		4 & 9823.53 & 0.281 & 2760.411 \\ \hline

		
	\end{tabular}	
		\caption{Wyniki pomiarów i obliczeń dla aluminium (z 4 pomiarów)}
		\label{Tabela2}	
	\end{table}

	Prędkość średnia fali dla aluminium:
	\begin{center}
		$\upsilon_{sr} = 2749.851$ \textrm{[m/s]}
	\end{center}
	
	Z podanych wartości możemy obliczyć wartość modułu Young'a ze wzoru (4) dla aluminium:
	\begin{center}
		\centering$E = 20.570$ [GPa]
	\end{center}

	%------------------------------------------------------------------------------------------------------
	
	\begin{table}[h!]
	\centering
		\begin{tabular}{|c|c|c|c|}
		
		\cline{1-2}
		\multicolumn{2}{|l|}{\begin{tabular}{c} \textrm{Analiza Mosiądzu} $l = 0.998$ [m]\end{tabular}} & \multicolumn{2}{c}{}\\
		\hline
		\textrm{Nr harmonicznej} & \textrm{Częstotliwość} $f$ \textrm{[Hz]} & \textrm{Długość fali} $ \lambda$ \textrm{[m]} & \textrm{Prędkość fali} $\upsilon$ \textrm{[m/s]}  \\ 
		\hline
		1 & 1776.47 & 1.996 & 3545.834 \\ \hline
		2  & 3458.82 & 0.998 & 3451.902 \\ \hline
		3 & 5152.94 & 0.665 & 3426.705 \\ \hline
		4 & 6835.29 & 0.499 & 3410.810 \\ \hline
		5 & 8653.53 & 0.399 & 3452.758 \\ \hline
		6 & 10294.12 & 0.333 & 3427.942 \\ \hline
		
	\end{tabular}	
	\caption{Wyniki obliczeń dla mosiądzu}
	\label{Tabela3}	
	\end{table}
	
	\begin{flushleft}
		Prędkość średnia fali dla mosiądzu:
	\end{flushleft}
	\begin{center}
		$\upsilon_{sr} = 3452.659$ \textrm{[m/s]}
	\end{center}
	
	\begin{flushleft}
	Z podanych wartości możemy obliczyć wartość modułu Young'a ze wzoru (4) dla mosiądzu:
	\end{flushleft}
	\begin{center}
		\centering$E = 100.152$ [GPa]
	\end{center}

	%----------------------------------------------------------------------------------------------------------

	\begin{table}[h!]
	\centering
	\begin{tabular}{|c|c|c|c|}
		
		\cline{1-2}
		\multicolumn{2}{|l|}{\begin{tabular}{c} \textrm{Analiza Stali nr 1} $l = 1.802$ [m]\end{tabular}} & \multicolumn{2}{c}{}\\
		\hline
		\textrm{Nr harmonicznej} & \textrm{Częstotliwość} $f$ \textrm{[Hz]} & \textrm{Długość fali} $ \lambda$ \textrm{[m]} & \textrm{Prędkość fali} $\upsilon$ \textrm{[m/s]}  \\ 
		\hline
		1 & 1400.00 & 3.604 & 5045.600 \\ \hline
		2 & 2905.88 & 1.802 & 5236.396 \\ \hline
		3 & 4305.88 & 1.201 & 5171.362 \\ \hline
		4 & 6670.59 & 0.901 & 6010.202 \\ \hline
		5 & 7905.88 & 0.721 & 5700.139 \\ \hline
		6 & 8623.53 & 0.601 & 5182.742 \\ \hline
		
	\end{tabular}	
	\caption{Wyniki obliczeń dla stali w kształcie prostopadłościanu}
	\label{Tabela4}	
\end{table}

\begin{flushleft}
	Prędkość średnia fali dla stali nr 1:
\end{flushleft}
\begin{center}
	$\upsilon_{sr} = 5391.074$ \textrm{[m/s]}
\end{center}

\begin{flushleft}
	Z podanych wartości możemy obliczyć wartość modułu Young'a ze wzoru (4) dla stali w kształcie prostopadłościanu:
\end{flushleft}
\begin{center}
	\centering$E = 218.836$ [GPa]
\end{center}

	%----------------------------------------------------------------------------------------------------------	
		
	\begin{table}[h!]
	\centering
	\begin{tabular}{|c|c|c|c|}
		
		\cline{1-2}
		\multicolumn{2}{|l|}{\begin{tabular}{c} \textrm{Analiza Stali nr 2} $l = 1.800$ [m]\end{tabular}} & \multicolumn{2}{c}{}\\
		\hline
		\textrm{Nr harmonicznej} & \textrm{Częstotliwość} $f$ \textrm{[Hz]} & \textrm{Długość fali} $ \lambda$ \textrm{[m]} & \textrm{Prędkość fali} $\upsilon$ \textrm{[m/s]}  \\ 
		\hline
		1 & 1400.00 & 3.600 & 5040.000 \\ \hline
		2 & 2905.88 & 1.800 & 5230.584 \\ \hline
		3 & 4305.88 & 1.200 & 5167.056 \\ \hline
		4 & 5811.76 & 0.900 & 5230.584 \\ \hline
		5 & 7211.78 & 0.720 & 5192.482 \\ \hline
		6 & 8623.53 & 0.600 & 5174.118 \\ \hline
		
	\end{tabular}	
	\caption{Wyniki obliczeń dla stali w kształcie prostopadłościanu}
	\label{Tabela5}	
\end{table}

\begin{flushleft}
	Prędkość średnia fali dla stali nr 2:
\end{flushleft}
\begin{center}
	$\upsilon_{sr} = 5172.471$ \textrm{[m/s]}
\end{center}

\begin{flushleft}
	Z podanych wartości możemy obliczyć wartość modułu Young'a ze wzoru (4) dla stali w kształcie walca:
\end{flushleft}
\begin{center}
	\centering$E = 207.119$ [GPa]
\end{center}	
	
	%----------------------------------------------------------------------------------------------------------	
	
\begin{table}[h!]
	\centering
	\begin{tabular}{|c|c|c|c|}
		
		\cline{1-2}
		\multicolumn{2}{|l|}{\begin{tabular}{c} \textrm{Analiza Żeliwa Szarego} $l = 1.800$ [m]\end{tabular}} & \multicolumn{2}{c}{}\\
		\hline
		\textrm{Nr harmonicznej} & \textrm{Częstotliwość} $f$ \textrm{[Hz]} & \textrm{Długość fali} $ \lambda$ \textrm{[m]} & \textrm{Prędkość fali} $\upsilon$ \textrm{[m/s]}  \\ 
		\hline
		1 & 1023.53 & 3.600 & 3684.708 \\ \hline
		2 & 2058.82 & 1.800 & 3705.876 \\ \hline
		3 & 3082.35 & 1.200 & 3698.820 \\ \hline
		4 & 4117.65 & 0.900 & 3705.885 \\ \hline
		5 & 5152.94 & 0.720 & 3710.117 \\ \hline
		6 & 6176.47 & 0.600 & 3705.882 \\ \hline
		
	\end{tabular}	
	\caption{Wyniki obliczeń dla żeliwa szarego}
	\label{Tabela6}	
\end{table}

	\newpage
	\begin{flushleft}
		Prędkość średnia fali dla żeliwa szarego:
	\end{flushleft}
	\begin{center}
		$\upsilon_{sr} = 3701.881$ \textrm{[m/s]}
	\end{center}
	
	\begin{flushleft}
		Z podanych wartości możemy obliczyć wartość modułu Young'a ze wzoru (4) dla żeliwa szarego w kształcie walca otwartego:
	\end{flushleft}
	
	\begin{center}
		\centering$E = 95.114$ [GPa]
	\end{center}	
	
	%----------------------------------------------------------------------------------------------------------
	
	\subsection{Analiza niepewności}
	
	Mamy do czynienia z niepewnością typu B, ponieważ pomiar był wykonywany tylko raz. Znana jest dokładność przyrządów mierniczych równa działką elementarnym, stąd wnioskujemy że dokładności pomiarów są równe ich wartością, dlatego:
	
	\subsubsection{Niepewności pomiaru długości prętów (miarka w rolce)}
	\begin{equation}
	u(l) = \textrm{działka elementarna} = 1 \textrm{ [mm]}
	\end{equation}
	
	\subsubsection{Niepewności pomiaru promienia/szerokości prętów (suwmiarka)}
		\begin{equation}
	u(r) = \textrm{działka elementarna} = 0.1 \textrm{ [mm]}
	\end{equation}
	
	\subsubsection{Niepewności pomiaru masy prętów (waga RADWAG WTB 200)}
	\begin{equation}
	u(m) = \textrm{działka elementarna} = 1 \textrm{ [g]}
	\end{equation}
	
	\subsubsection{Niepewności pomiaru zarejestrowanej częstotliwości (komputer z podłączonym mikrofonem)}
	\begin{equation}
	u(f) =  25 \textrm{ [Hz]}
	\end{equation}

	\begin{flushleft}
		\newpage Niepewność obliczeń wykonanych na podstawie pozyskanych danych możemy przedstawić za pomocą wzorów:
	\end{flushleft}

	\subsubsection{Niepewność gęstości}
	
	$$ u(\rho)=\sqrt{\bigg(\frac{\partial \rho}{\partial m}u(m)\bigg)^2+\bigg(\frac{\partial \rho}{\partial l}u(l)\bigg)^2+\bigg(\frac{\partial \rho}{\partial r}u(r)\bigg)^2}$$
	
\begin{table}[h!]
	\centering
	\begin{tabular}{|c|c|c|c|c|c|}
		\hline
		\textrm{Materiał} & \textrm{Aluminium} & \textrm{Mosiądz} & \textrm{Stal nr 1} & \textrm{Stal nr 2}  & \textrm{Żeliwo szare}  \\ \hline
		\textrm{Niepewność gęstości $u(\rho)$ [kg/m$^3$] } & 236.206 & 773.854 & 105.059 & 303.677 & 562.319 \\ \hline
	\end{tabular}	
\caption{Wyniki obliczeń niepewności gęstości}
\label{Tabela7}	
\end{table}

	\subsubsection{Niepewność długości fali}	
	$$ u(\lambda)=\sqrt{\bigg(\frac{2}{n}u(l)\bigg)^2}$$
	
	\subsubsection{Niepewność prędkości fali}
	$$ u(v)=\sqrt{\bigg(\frac{\partial v}{\partial f}u(f)\bigg)^2+\bigg(\frac{\partial v}{\partial \lambda}u(\lambda)\bigg)^2}=\sqrt{\bigg(\lambda u(f)\bigg)^2+\bigg(f u(\lambda)\bigg)^2}$$
	
	\begin{table}[h!]
		\centering
		\begin{tabular}{|c|c|c|c|c|c|}
			\hline
			\textrm{Materiał} & \textrm{Aluminium} & \textrm{Mosiądz} & \textrm{Stal nr 1} & \textrm{Stal nr 2}  & \textrm{Żeliwo szare}  \\ \hline
			\textrm{Niepewność prędkości fali $u(v)$ [m/s] } & 28.461 & 50.026 & 90.143 & 90.044 & 90.023 \\ \hline
		\end{tabular}	
		\caption{Wyniki obliczeń niepewności prędkości fali}
		\label{Tabela8}	
	\end{table}
	
	\subsubsection{Niepewność modułu Young'a}
	$$ u(E)=\sqrt{\bigg(\frac{\partial E}{\partial \rho}u(\rho)\bigg)^2+\bigg(\frac{\partial E}{\partial v}u(v)\bigg)^2} =
	\sqrt{\bigg(v^2 u(\rho)\bigg)^2+\bigg(2 \rho v u(v)\bigg)^2}$$

	\begin{table}[h!]
	\centering
	\begin{tabular}{|c|c|c|c|c|c|}
		\hline
		\textrm{Materiał} & \textrm{Aluminium} & \textrm{Mosiądz} & \textrm{Stal nr 1} & \textrm{Stal nr 2}  & \textrm{Żeliwo szare}  \\ \hline
		\textrm{Niepewność modułu Young'a $u(E)$ [GPa] } & 1.836 & 9.6707 & 7.929 & 10.863 & 8.987 \\ \hline
	\end{tabular}	
	\caption{Wyniki obliczeń niepewności modułu Young'a}
	\label{Tabela9}	
	\end{table}
	
%--------------------------------------------------------------------------------------------------------------
\newpage
\section{Podsumowanie otrzymanych wyników końcowych}
	
\begin{table}[h!]
	\centering
	\begin{tabular}{|c|c|c|}
		
		\hline
		Materiał & \begin{tabular}{c} Wartość tabelaryczna  \\ \textrm{[GPa]}  \end{tabular}  & \begin{tabular}{c} Wartość wyznaczona  \\  \textrm{[GPa]}  \end{tabular} \\
		\hline
		Aluminium & 69 & 20.570 \\[0.3ex] \hline
		Mosiądz & 100  & 100.152 \\[0.3ex] \hline
		Stal nr 1 & 205-210 & 218.836 \\[0.3ex] \hline
		Stal nr 2 & 205-210  & 207.119 \\[0.3ex] \hline
		Żeliwo szare & 125 & 95.114 \\[0.3ex] \hline

	\end{tabular}
	\caption{Porównanie wartości tabelaryczych z otrzymanymi podczas wykonywania doświadczenia}
	\label{Tabela10}	
\end{table}

\flushleft
W powyższej tabeli porównano wartości tabelaryczne z wartości wyznaczonymi. Przy rozszerzonej niepewności pomiaru (dla $k = 2$), wyniki mosiądzu oraz stali nr 1 i 2 zgadzają się z wartościami tabelarycznymi. Wartości otrzymane dla aluminium i żeliwa szarego znacznie odbiegają od wartości tabelarycznej. Spowodowane jest to wadliwymi próbkami pomiarowymi (widocznie zanieczyszczonymi oraz odgiętymi), niedokładnymi pomiarami. Wynik końcowy aluminium uznajemy ostatecznie za błąd gruby dlatego uznajemy go za błędny.
	

%--------------------------------------------------------------------------------------------------------------

	\section{Wnioski}
		Przeprowadzone doświadczenie pozwoliło nam nabrać doświadczenia w pomiarach związanych z dźwiękiem, pomiarów wymiarów fizycznych ciał oraz finalnie zbadanie modułu Younga dla różnych materiałów. Z racji, że dwa z mierzonych ciał wykonane były z tego samego materiału (stal), lecz o różnym kształcie, dowiedliśmy się, że (w granicy dopuszczalnego błędu) kształt ciała nie ma wpływu na wartość modułu Younga. Niepewności pomiarowe wywodzą się z hałasu otoczenia w sali laboratoryjnej oraz zastosowania mikrofonu nie najwyższej jakości. Dodatkowo mogą one wynikać z problemów z zatrzymaniem odczytu programu Zelscope w momencie, w którym najlepiej widać w jakich częstotliwościach występują kolejne harmoniczne widma fali dźwiękowej, co prowadziło do wielokrotnego powtarzania pomiaru. Wszelkie niedoskonałości czy zabrudzenia materiałów również miały wpływ na wynik końcowy. Na szczególną uwagę zasługuje badane aluminium, którego wynik zdecydowanie odbiega od wartości tablicowych i uznajemy go za niewłaściwy. Sam pręt wyróżniał się wśród wszystkich mierzonych swoją krzywizną, miał naklejoną nalepkę na jednym z końców oraz jego przekrój nie tworzył idealnego koła. Aby osiągnąć najdokładniejszy wynik należy zadbać o jak najlepszą jakość badanych ciał.
	
\end{document}