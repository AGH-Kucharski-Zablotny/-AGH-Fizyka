%!Mode:: "TeX:UTF-8"
\documentclass[a4paper,12pts]{article}

\usepackage[polish]{babel}
\usepackage[utf8]{inputenc}
\usepackage{fontspec}
\setmainfont{Calibri}

\linespread{1.15}

\usepackage{caption}
\captionsetup{%
	font={footnotesize},
	labelfont={bf}
}

\usepackage{anysize}
\usepackage{geometry}

\usepackage{graphicx}

% Plik szablonowy do wykorzystania pózniej - nie zmieniaj go!

\begin{document}
	\thispagestyle{empty}
	\begin{flushleft}
		Wydział Elektrotechniki, Automatyki, Informatyki i Inżynierii Biomedycznej \\
		Informatyka, rok II \\
		Zespół numer 3 \\
		Piotr Kucharski \\
		Dominik Zabłotny \\
		\vspace*{\fill}
		%-----------NUMER CWICZENIA--------%
		{\large \textbf{Sprawozdanie z ćwiczenia nr 35} } \\
		%-----------TEMAT ĆWICZENIA--------%
		Elektroliza		
		\vfill	
		%-----------DATA-------------%
		8 listopada 2017r
	\end{flushleft}
	
	\newpage
	
%--------------------------------------------------------------------------------------------------------------
	
	\section{Wstęp}
	
		\subsection{Cel ćwiczenia}
		Wyznaczenie stałej Faradaya oraz równoważnika elektrochemicznego miedzi metodą elektrolizy.
	
	%----------------------------------------------------------------------------------------------------------
	
		\subsection{Wprowadzenie teoretyczne}
	
			\subsubsection{Dysocjacja elektrolityczna}
			Proces rozpadu cząstek związków chemicznych na jony pod wpływem rozpuszczalnika nazywamy dysocjacją elektrolityczną. Zjawisku temu podlegają związki z wiązaniami jonowymi oraz bardzo silnie spolaryzowane kowalencyjnie. Jest to proces odwracalny, wiele związków ulega autodysocjacji w stanie ciekłym i gazowym (np. woda).
			
			\subsubsection{Elektroliza}
			Proces zmiany struktury chemicznej substancji - a dokładniej procesy rozkładu, zwykle zachodzące pod wpływem przyłożonego napięcia elektrycznego. Do pojęcia elektrolizy zalicza wiele zjawisk, takich jak dysocjacja elektrolityczna, transport jonów do elektrod, wtórne przemiany jonów na elektrodach i inne. Po przyłożeniu odpowiedniego dla danej substancji napięcia prądu dochodzi do wymuszonej wędrówki jonów do elektrod zanurzonych w  substancji - odpowiednio do katody dążą kationy a do anody dążą aniony. Wynikiem elektrolizy jest zamiana w obojętne elektrycznie związki chemiczne lub pierwiastki. Masa substancji wydzielonej na elektrodzie w wuniku elektrolizy jest wprost proporcjonalna do ładunku przepływającego przez elektrolit (I. prawo Faradaya)
			\begin{equation}
				m = Itk
			\end{equation}
			gdzie $I$ to natężenie prądu, $t$ to czas a $k$ to równoważnik eletrochemiczny.
			
			\subsubsection{Masa molowa}
			Masa jednego mola substancji chemicznej wyrażana jednostką $\frac{kg}{mol}$
			
			\subsubsection{Wartościowość}
			Cecha pierwiastków chemicznych mówiąca o liczbie wiązań chemicznych, którymi pierwiastek lub jon może łączyć się z innymi. Dany pierwiastek może posiadać wiele wartościowości zależnych od stopnia utlenienia.
			
			\subsubsection{Jony}
			Jony to atomy lub grupy atomów połączonych wiązaniami chemicznymi, która ma niedomiar protonów (wówczas nazywamy je anionami) lub nadmiar protonów w stosunku do elektronów (wówczas nazywamy je kationami).
			
			\subsubsection{Katoda}
			Elektroda, przez którą z urządzenia wypływa prąd elektryczny. W urządzeniach elektrycznych katoda jest elektordą ujemną, w źródłach prądu jest elektrodą dodatnią.
			
			\subsubsection{Anoda}
			Elektroda przeciwna do katody, przez nią prąd "wpływa" do urządzenia. W odbiornikach jest to elektroda dodatnia a w źródłach prądu ujemna.
			
			\subsubsection{I. prawo elektrolizy Faradaya (1834r.)}
			Masa substancji wydzielonej podczas elektrolizy jest proporcjonalna do ładunku, który przepłynął przez elektrolit.
			\begin{equation}
				m = qk = Itk
			\end{equation}
			gdzie $k$ to równoważnik elektrochemiczny, $q$ to ładunek elektryczny, $I$ to natężenie prądu elektrycznego oraz $t$ to czas elektrolizy.
			
			\subsubsection{II. prawo elektrolizy Faradaya (1834r.)}
			Ładunek q potrzebny do wydzielenia lub wchłonięcia masy m jest dany zależnością:
			\begin{equation}
				q = \frac{Fmz}{M}
			\end{equation}
			gdzie $F$ to stała Faradaya wrażana jednostką $\left[\frac{C}{mol}\right]$, $z$ to ładunek jonu bez jednostki oraz $M$ to masa molowa jonu wyrażona jednostką $\left[\frac{g}{mol}\right]$
			
			\subsubsection{Stała Faradaya}
			Stała Faradaya wyraża ładunek elektryuczny przypadający na jeden mol eletronów oraz określa się ją wzorem
			\begin{equation}
				F = N_A e
			\end{equation}
			gdzie $N_A$ to stała Avogadra ($N_A \approx $) oraz $e$ to ładunek elektronu		
	
	%----------------------------------------------------------------------------------------------------------
	
	
	
%--------------------------------------------------------------------------------------------------------------
	
	\section{Wykonanie ćwiczenia}
	
%--------------------------------------------------------------------------------------------------------------
	
	\section{Opracowanie danych pomiarowych}
	
	%----------------------------------------------------------------------------------------------------------	
	
	\subsection{Analiza niepewności}
	
%--------------------------------------------------------------------------------------------------------------

	\section{Podsumowanie}

%--------------------------------------------------------------------------------------------------------------
	
	\section{Wnioski}

\end{document}